
%% bare_jrnl.tex
%% V1.4
%% 2012/12/27
%% by Michael Shell
%% see http://www.michaelshell.org/
%% for current contact information.

\documentclass[journal]{IEEEtran}


\begin{document}
%
% paper title
% can use linebreaks \\ within to get better formatting as desired
% Do not put math or special symbols in the title.
\title{Tensegrity Based Probes for Planetary Exploration:\\ Entry, Descent and Landing (EDL) and Surface\\ Mobility Analysis}
%
%
% author names and IEEE memberships
% note positions of commas and nonbreaking spaces ( ~ ) LaTeX will not break
% a structure at a ~ so this keeps an author's name from being broken across
% two lines.
% use \thanks{} to gain access to the first footnote area
% a separate \thanks must be used for each paragraph as LaTeX2e's \thanks
% was not built to handle multiple paragraphs
%

\author{George Gorospe, Jonathan Bruce, Atil Iscen, Adrian Agogino, David Atkinson, Vytas SunSpiral}


% The paper headers
\markboth{Interplanetary Probe Journal}%

% make the title area
\maketitle

% As a general rule, do not put math, special symbols or citations
% in the abstract or keywords.
\begin{abstract}
Low-cost yet highly capable planetary entry probes have the potential to revolutionize science return even at today’s modest space budgets. However, it is nearly impossible to imagine such probes being built with traditional rigid robotics and traditional landing systems due to the high weight, complexity, and high mission control costs of these systems.  A novel probe concept based on tensegrity structures can overcome many of these limitations. The core tensegrity structure is entirely formed out of tensile components (cables) and compression components (light-weight rods) to create a tensile network. Our concept is to develop a tensegrity probe in which the tensile network can be actively controlled to enable compact stowage for launch followed by deployment upon landing. Due to their natural compliance and structural force distribution properties, tensegrity probes can safely absorb significant impact forces, enabling high speed Entry, Descent, and Landing (EDL) scenarios where the probe itself acts much like an airbag.  However, unlike an airbag which must be discarded after a single use, the tensegrity probe can actively control its shape to provide compliant rolling mobility while still maintaining its ability to safely absorb impact shocks that might occur during exploration.  This combination of functions from a single structure enables compact and light-weight planetary exploration missions with the capabilities of traditional wheeled rovers, but with the mass and cost similar or less than a stationary probe.
In this paper we evaluate tensegrity probes on the basis of the EDL phase performance of the probe in the context of a mission to Titan.  Titan’s atmosphere, stable bodies of surface liquid and complex organics make it one of the most complex and Earth-like environments in the solar system. This tensegrity probe mission will build on the science returns from Huygens and answer many of the new and unresolved questions surrounding Titan’s ongoing organic processes, geologic history, atmosphere, and surface-atmosphere interactions. Upon arrival at Titan, the tightly packed tensegrity probes will separate from the spacecraft, either individually or simultaneously to target specific regions or environments on the surface, and begin atmospheric descent. Without requiring parachutes, each probe is projected to impact the surface at about 11 m/s, absorbing and distributing impact stresses while protecting its science payload. As the tensegrity probes bounce, roll, and finally come to a rest on the surface of Titan, the actuated tensegrity structure will then begin to function as the primary mobility system for these mobile probes. Once on the surface, a notional science payload containing an atmospheric package, an analytical chemistry package, and an imaging package can begin the probes’ science mission. Since each probe is equipped with a science payload, mission success is not entirely dependent on the survival of a single probe, and the deployment of only a few probes enables the distributed exploration of a planetary surface. 
An individual tensegrity probe has a significantly lower EDL hardware overhead and increased science payload mass percentage when compared to similar planetary surface probe missions, increasing the potential for science return. Compared with the Mars Exploration Rovers (MER) and Mars Science Laboratory (MSL) which employed complex EDL systems to support roving vehicles with a science payload mass fraction of 1.6% for MER and 3.33% for MSL, a tensegrity probe may be able to operate on Titan as a mobile probe with a science payload mass fraction of up to 50% due to the dual use of the tensegrity structure as for both EDL and surface mobility. For applications on other targets such as Mars, and given an EDL scenario similar to MER with a 100 km/h (27.7 m/s) impact velocity, a tensegrity probe may be able to survive EDL and carry a higher percentage science payload by mass
A key quality of these probes is their ability to shock-protect core systems (science payloads, avionics, and actuators), even when landing at relatively high speeds. We will discuss analytical, simulated, and experimental results of terminal velocity landing events to show how the impact forces are distributed through the structure and reduce the internal stress at any one point within the probe.  We will then discuss how these results are guiding engineering choices for actuation and material properties of the tensegrity probe and our ongoing efforts to develop and test physical prototypes.  Given the multi-function use of the tensegrity probe, we will then describe how the structure can also enable mobility for exploration of a landing site. Finally, the tensegrity probe is directly compared to other EDL and mobility solutions with a focus on cost, complexity, delivered science payload, and robustness.

\end{abstract}

% Note that keywords are not normally used for peerreview papers.
\begin{IEEEkeywords}
IEEEtran, journal, \LaTeX, paper, template.
\end{IEEEkeywords}






% For peer review papers, you can put extra information on the cover
% page as needed:
% \ifCLASSOPTIONpeerreview
% \begin{center} \bfseries EDICS Category: 3-BBND \end{center}
% \fi
%
% For peerreview papers, this IEEEtran command inserts a page break and
% creates the second title. It will be ignored for other modes.
\IEEEpeerreviewmaketitle



\section{Introduction}
% The very first letter is a 2 line initial drop letter followed
% by the rest of the first word in caps.
% 
% form to use if the first word consists of a single letter:
% \IEEEPARstart{A}{demo} file is ....
% 
% form to use if you need the single drop letter followed by
% normal text (unknown if ever used by IEEE):
% \IEEEPARstart{A}{}demo file is ....
% 
% Some journals put the first two words in caps:
% \IEEEPARstart{T}{his demo} file is ....
% 
% Here we have the typical use of a "T" for an initial drop letter
% and "HIS" in caps to complete the first word.
\IEEEPARstart{T}{his} demo file is intended to serve as a ``starter file''
for IEEE journal papers produced under \LaTeX\ using
IEEEtran.cls version 1.8 and later.
% You must have at least 2 lines in the paragraph with the drop letter
% (should never be an issue)
I wish you the best of success.

\hfill mds
 
\hfill December 27, 2012

\subsection{Subsection Heading Here}
Subsection text here.

% needed in second column of first page if using \IEEEpubid
%\IEEEpubidadjcol

\subsubsection{Subsubsection Heading Here}
Subsubsection text here.


% An example of a floating figure using the graphicx package.
% Note that \label must occur AFTER (or within) \caption.
% For figures, \caption should occur after the \includegraphics.
% Note that IEEEtran v1.7 and later has special internal code that
% is designed to preserve the operation of \label within \caption
% even when the captionsoff option is in effect. However, because
% of issues like this, it may be the safest practice to put all your
% \label just after \caption rather than within \caption{}.
%
% Reminder: the "draftcls" or "draftclsnofoot", not "draft", class
% option should be used if it is desired that the figures are to be
% displayed while in draft mode.
%
%\begin{figure}[!t]
%\centering
%\includegraphics[width=2.5in]{myfigure}
% where an .eps filename suffix will be assumed under latex, 
% and a .pdf suffix will be assumed for pdflatex; or what has been declared
% via \DeclareGraphicsExtensions.
%\caption{Simulation Results.}
%\label{fig_sim}
%\end{figure}

% Note that IEEE typically puts floats only at the top, even when this
% results in a large percentage of a column being occupied by floats.


% An example of a double column floating figure using two subfigures.
% (The subfig.sty package must be loaded for this to work.)
% The subfigure \label commands are set within each subfloat command,
% and the \label for the overall figure must come after \caption.
% \hfil is used as a separator to get equal spacing.
% Watch out that the combined width of all the subfigures on a 
% line do not exceed the text width or a line break will occur.
%
%\begin{figure*}[!t]
%\centering
%\subfloat[Case I]{\includegraphics[width=2.5in]{box}%
%\label{fig_first_case}}
%\hfil
%\subfloat[Case II]{\includegraphics[width=2.5in]{box}%
%\label{fig_second_case}}
%\caption{Simulation results.}
%\label{fig_sim}
%\end{figure*}
%
% Note that often IEEE papers with subfigures do not employ subfigure
% captions (using the optional argument to \subfloat[]), but instead will
% reference/describe all of them (a), (b), etc., within the main caption.


% An example of a floating table. Note that, for IEEE style tables, the 
% \caption command should come BEFORE the table. Table text will default to
% \footnotesize as IEEE normally uses this smaller font for tables.
% The \label must come after \caption as always.
%
%\begin{table}[!t]
%% increase table row spacing, adjust to taste
%\renewcommand{\arraystretch}{1.3}
% if using array.sty, it might be a good idea to tweak the value of
% \extrarowheight as needed to properly center the text within the cells
%\caption{An Example of a Table}
%\label{table_example}
%\centering
%% Some packages, such as MDW tools, offer better commands for making tables
%% than the plain LaTeX2e tabular which is used here.
%\begin{tabular}{|c||c|}
%\hline
%One & Two\\
%\hline
%Three & Four\\
%\hline
%\end{tabular}
%\end{table}


% Note that IEEE does not put floats in the very first column - or typically
% anywhere on the first page for that matter. Also, in-text middle ("here")
% positioning is not used. Most IEEE journals use top floats exclusively.
% Note that, LaTeX2e, unlike IEEE journals, places footnotes above bottom
% floats. This can be corrected via the \fnbelowfloat command of the
% stfloats package.



\section{Conclusion}
The conclusion goes here.





% if have a single appendix:
%\appendix[Proof of the Zonklar Equations]
% or
%\appendix  % for no appendix heading
% do not use \section anymore after \appendix, only \section*
% is possibly needed

% use appendices with more than one appendix
% then use \section to start each appendix
% you must declare a \section before using any
% \subsection or using \label (\appendices by itself
% starts a section numbered zero.)
%


\appendices
\section{Proof of the First Zonklar Equation}
Appendix one text goes here.

% you can choose not to have a title for an appendix
% if you want by leaving the argument blank
\section{}
Appendix two text goes here.


% use section* for acknowledgement
\section*{Acknowledgment}


The authors would like to thank...


% Can use something like this to put references on a page
% by themselves when using endfloat and the captionsoff option.
\ifCLASSOPTIONcaptionsoff
  \newpage
\fi



% trigger a \newpage just before the given reference
% number - used to balance the columns on the last page
% adjust value as needed - may need to be readjusted if
% the document is modified later
%\IEEEtriggeratref{8}
% The "triggered" command can be changed if desired:
%\IEEEtriggercmd{\enlargethispage{-5in}}

% references section

% can use a bibliography generated by BibTeX as a .bbl file
% BibTeX documentation can be easily obtained at:
% http://www.ctan.org/tex-archive/biblio/bibtex/contrib/doc/
% The IEEEtran BibTeX style support page is at:
% http://www.michaelshell.org/tex/ieeetran/bibtex/
%\bibliographystyle{IEEEtran}
% argument is your BibTeX string definitions and bibliography database(s)
%\bibliography{IEEEabrv,../bib/paper}
%
% <OR> manually copy in the resultant .bbl file
% set second argument of \begin to the number of references
% (used to reserve space for the reference number labels box)
\begin{thebibliography}{1}

\bibitem{IEEEhowto:kopka}
H.~Kopka and P.~W. Daly, \emph{A Guide to \LaTeX}, 3rd~ed.\hskip 1em plus
  0.5em minus 0.4em\relax Harlow, England: Addison-Wesley, 1999.

\end{thebibliography}

% biography section
% 
% If you have an EPS/PDF photo (graphicx package needed) extra braces are
% needed around the contents of the optional argument to biography to prevent
% the LaTeX parser from getting confused when it sees the complicated
% \includegraphics command within an optional argument. (You could create
% your own custom macro containing the \includegraphics command to make things
% simpler here.)
%\begin{IEEEbiography}[{\includegraphics[width=1in,height=1.25in,clip,keepaspectratio]{mshell}}]{Michael Shell}
% or if you just want to reserve a space for a photo:

\begin{IEEEbiography}{Michael Shell}
Biography text here.
\end{IEEEbiography}

% if you will not have a photo at all:
\begin{IEEEbiographynophoto}{John Doe}
Biography text here.
\end{IEEEbiographynophoto}

% insert where needed to balance the two columns on the last page with
% biographies
%\newpage

\begin{IEEEbiographynophoto}{Jane Doe}
Biography text here.
\end{IEEEbiographynophoto}

% You can push biographies down or up by placing
% a \vfill before or after them. The appropriate
% use of \vfill depends on what kind of text is
% on the last page and whether or not the columns
% are being equalized.

%\vfill

% Can be used to pull up biographies so that the bottom of the last one
% is flush with the other column.
%\enlargethispage{-5in}



% that's all folks
\end{document}


